% Options for packages loaded elsewhere
\PassOptionsToPackage{unicode}{hyperref}
\PassOptionsToPackage{hyphens}{url}
%
\documentclass[
]{article}
\usepackage{amsmath,amssymb}
\usepackage{iftex}
\ifPDFTeX
  \usepackage[T1]{fontenc}
  \usepackage[utf8]{inputenc}
  \usepackage{textcomp} % provide euro and other symbols
\else % if luatex or xetex
  \usepackage{unicode-math} % this also loads fontspec
  \defaultfontfeatures{Scale=MatchLowercase}
  \defaultfontfeatures[\rmfamily]{Ligatures=TeX,Scale=1}
\fi
\usepackage{lmodern}
\ifPDFTeX\else
  % xetex/luatex font selection
\fi
% Use upquote if available, for straight quotes in verbatim environments
\IfFileExists{upquote.sty}{\usepackage{upquote}}{}
\IfFileExists{microtype.sty}{% use microtype if available
  \usepackage[]{microtype}
  \UseMicrotypeSet[protrusion]{basicmath} % disable protrusion for tt fonts
}{}
\makeatletter
\@ifundefined{KOMAClassName}{% if non-KOMA class
  \IfFileExists{parskip.sty}{%
    \usepackage{parskip}
  }{% else
    \setlength{\parindent}{0pt}
    \setlength{\parskip}{6pt plus 2pt minus 1pt}}
}{% if KOMA class
  \KOMAoptions{parskip=half}}
\makeatother
\usepackage{xcolor}
\usepackage[margin=1in]{geometry}
\usepackage{color}
\usepackage{fancyvrb}
\newcommand{\VerbBar}{|}
\newcommand{\VERB}{\Verb[commandchars=\\\{\}]}
\DefineVerbatimEnvironment{Highlighting}{Verbatim}{commandchars=\\\{\}}
% Add ',fontsize=\small' for more characters per line
\usepackage{framed}
\definecolor{shadecolor}{RGB}{248,248,248}
\newenvironment{Shaded}{\begin{snugshade}}{\end{snugshade}}
\newcommand{\AlertTok}[1]{\textcolor[rgb]{0.94,0.16,0.16}{#1}}
\newcommand{\AnnotationTok}[1]{\textcolor[rgb]{0.56,0.35,0.01}{\textbf{\textit{#1}}}}
\newcommand{\AttributeTok}[1]{\textcolor[rgb]{0.13,0.29,0.53}{#1}}
\newcommand{\BaseNTok}[1]{\textcolor[rgb]{0.00,0.00,0.81}{#1}}
\newcommand{\BuiltInTok}[1]{#1}
\newcommand{\CharTok}[1]{\textcolor[rgb]{0.31,0.60,0.02}{#1}}
\newcommand{\CommentTok}[1]{\textcolor[rgb]{0.56,0.35,0.01}{\textit{#1}}}
\newcommand{\CommentVarTok}[1]{\textcolor[rgb]{0.56,0.35,0.01}{\textbf{\textit{#1}}}}
\newcommand{\ConstantTok}[1]{\textcolor[rgb]{0.56,0.35,0.01}{#1}}
\newcommand{\ControlFlowTok}[1]{\textcolor[rgb]{0.13,0.29,0.53}{\textbf{#1}}}
\newcommand{\DataTypeTok}[1]{\textcolor[rgb]{0.13,0.29,0.53}{#1}}
\newcommand{\DecValTok}[1]{\textcolor[rgb]{0.00,0.00,0.81}{#1}}
\newcommand{\DocumentationTok}[1]{\textcolor[rgb]{0.56,0.35,0.01}{\textbf{\textit{#1}}}}
\newcommand{\ErrorTok}[1]{\textcolor[rgb]{0.64,0.00,0.00}{\textbf{#1}}}
\newcommand{\ExtensionTok}[1]{#1}
\newcommand{\FloatTok}[1]{\textcolor[rgb]{0.00,0.00,0.81}{#1}}
\newcommand{\FunctionTok}[1]{\textcolor[rgb]{0.13,0.29,0.53}{\textbf{#1}}}
\newcommand{\ImportTok}[1]{#1}
\newcommand{\InformationTok}[1]{\textcolor[rgb]{0.56,0.35,0.01}{\textbf{\textit{#1}}}}
\newcommand{\KeywordTok}[1]{\textcolor[rgb]{0.13,0.29,0.53}{\textbf{#1}}}
\newcommand{\NormalTok}[1]{#1}
\newcommand{\OperatorTok}[1]{\textcolor[rgb]{0.81,0.36,0.00}{\textbf{#1}}}
\newcommand{\OtherTok}[1]{\textcolor[rgb]{0.56,0.35,0.01}{#1}}
\newcommand{\PreprocessorTok}[1]{\textcolor[rgb]{0.56,0.35,0.01}{\textit{#1}}}
\newcommand{\RegionMarkerTok}[1]{#1}
\newcommand{\SpecialCharTok}[1]{\textcolor[rgb]{0.81,0.36,0.00}{\textbf{#1}}}
\newcommand{\SpecialStringTok}[1]{\textcolor[rgb]{0.31,0.60,0.02}{#1}}
\newcommand{\StringTok}[1]{\textcolor[rgb]{0.31,0.60,0.02}{#1}}
\newcommand{\VariableTok}[1]{\textcolor[rgb]{0.00,0.00,0.00}{#1}}
\newcommand{\VerbatimStringTok}[1]{\textcolor[rgb]{0.31,0.60,0.02}{#1}}
\newcommand{\WarningTok}[1]{\textcolor[rgb]{0.56,0.35,0.01}{\textbf{\textit{#1}}}}
\usepackage{graphicx}
\makeatletter
\def\maxwidth{\ifdim\Gin@nat@width>\linewidth\linewidth\else\Gin@nat@width\fi}
\def\maxheight{\ifdim\Gin@nat@height>\textheight\textheight\else\Gin@nat@height\fi}
\makeatother
% Scale images if necessary, so that they will not overflow the page
% margins by default, and it is still possible to overwrite the defaults
% using explicit options in \includegraphics[width, height, ...]{}
\setkeys{Gin}{width=\maxwidth,height=\maxheight,keepaspectratio}
% Set default figure placement to htbp
\makeatletter
\def\fps@figure{htbp}
\makeatother
\setlength{\emergencystretch}{3em} % prevent overfull lines
\providecommand{\tightlist}{%
  \setlength{\itemsep}{0pt}\setlength{\parskip}{0pt}}
\setcounter{secnumdepth}{-\maxdimen} % remove section numbering
\ifLuaTeX
  \usepackage{selnolig}  % disable illegal ligatures
\fi
\IfFileExists{bookmark.sty}{\usepackage{bookmark}}{\usepackage{hyperref}}
\IfFileExists{xurl.sty}{\usepackage{xurl}}{} % add URL line breaks if available
\urlstyle{same}
\hypersetup{
  pdftitle={统计计算第三次作业报告},
  pdfauthor={林子开},
  hidelinks,
  pdfcreator={LaTeX via pandoc}}

\title{统计计算第三次作业报告}
\author{林子开}
\date{2023-10-13}

\begin{document}
\maketitle

\hypertarget{exercise-5.11}{%
\subsection{Exercise 5.11}\label{exercise-5.11}}

Assume that \(\hat{\theta_1}\) and \(\hat{\theta_2}\) are two unbiased
estimators of \(\theta\). Moreover, we suppose that they are two
distinct estimators such that \(Var(\theta_1-\theta_2)>0\), otherwise
\(c^*\) can be a arbitrary number.

The variance of \(\hat{\theta_c}=c\hat{\theta_1}+(1-c)\hat{\theta_2}\)
is: \[
\begin{align}
  Var(\hat{\theta_c})&= c^2Var(\hat{\theta_1}) + (1-c)^2Var(\hat{\theta_2})+2c(1-c)Cov(\hat{\theta_1},\hat{\theta_2}) \\
  &= \left[Var(\hat{\theta_1})+Var(\hat{\theta_2})-2Cov(\hat{\theta_1},\hat{\theta_2})\right]c^2
  +\left[-2Var(\hat{\theta_2})+2Cov(\hat{\theta_1},\hat{\theta_2})\right]c
  + Var(\hat{\theta_2})
\end{align}
\] Since we have the assumption that
\[Var(\theta_1-\theta_2) = \left[Var(\hat{\theta_1})+Var(\hat{\theta_2})-2Cov(\hat{\theta_1},\hat{\theta_2})\right]>0\]

we can obtain the optimal \(c^*\): \[ 
c^* = \frac{-Var(\hat{\theta_2})+Cov(\hat{\theta_1},\hat{\theta_2})}{Var(\hat{\theta_1})+Var(\hat{\theta_2})-2Cov(\hat{\theta_1},\hat{\theta_2})}
\]

\(\blacksquare\)

\hypertarget{estimating-the-rare-event-probability}{%
\subsection{Estimating the rare event
probability}\label{estimating-the-rare-event-probability}}

\hypertarget{the-sample-sizes-when-applying-smc-method}{%
\subsubsection{The sample sizes when applying SMC
method}\label{the-sample-sizes-when-applying-smc-method}}

\begin{Shaded}
\begin{Highlighting}[]
\FunctionTok{set.seed}\NormalTok{(}\DecValTok{123}\NormalTok{)}
\NormalTok{sampleSize.SMC }\OtherTok{=} \FunctionTok{c}\NormalTok{(}\DecValTok{3}\SpecialCharTok{:}\DecValTok{8}\NormalTok{)}
\NormalTok{result.SMC }\OtherTok{=} \FunctionTok{numeric}\NormalTok{(}\FunctionTok{length}\NormalTok{(sampleSize.SMC))}
\NormalTok{count }\OtherTok{=} \DecValTok{0}
\ControlFlowTok{for}\NormalTok{(n }\ControlFlowTok{in}\NormalTok{ sampleSize.SMC)\{}
\NormalTok{  count }\OtherTok{=}\NormalTok{ count }\SpecialCharTok{+} \DecValTok{1}
\NormalTok{  u }\OtherTok{=} \FunctionTok{rnorm}\NormalTok{(}\DecValTok{10}\SpecialCharTok{\^{}}\NormalTok{n)}
\NormalTok{  result.SMC[count] }\OtherTok{=} \FunctionTok{mean}\NormalTok{(u}\SpecialCharTok{\textgreater{}}\DecValTok{10}\NormalTok{)}
\NormalTok{\}}
\FunctionTok{rbind}\NormalTok{(sampleSize.SMC,result.SMC)}
\end{Highlighting}
\end{Shaded}

\begin{verbatim}
##                [,1] [,2] [,3] [,4] [,5] [,6]
## sampleSize.SMC    3    4    5    6    7    8
## result.SMC        0    0    0    0    0    0
\end{verbatim}

\hypertarget{the-sample-sizes-when-applying-importance-sampling}{%
\subsubsection{The sample sizes when applying importance
sampling}\label{the-sample-sizes-when-applying-importance-sampling}}

\begin{Shaded}
\begin{Highlighting}[]
\FunctionTok{set.seed}\NormalTok{(}\DecValTok{123}\NormalTok{)}
\NormalTok{sampleSize.ImpS }\OtherTok{=} \FunctionTok{c}\NormalTok{(}\DecValTok{3}\SpecialCharTok{:}\DecValTok{8}\NormalTok{)}
\NormalTok{result.ImpS }\OtherTok{=} \FunctionTok{numeric}\NormalTok{(}\FunctionTok{length}\NormalTok{(sampleSize.SMC))}
\NormalTok{count }\OtherTok{=} \DecValTok{0}
\ControlFlowTok{for}\NormalTok{(n }\ControlFlowTok{in}\NormalTok{ sampleSize.ImpS)\{}
\NormalTok{  count }\OtherTok{=}\NormalTok{ count }\SpecialCharTok{+} \DecValTok{1}
\NormalTok{  u }\OtherTok{=} \FunctionTok{rnorm}\NormalTok{(}\DecValTok{10}\SpecialCharTok{\^{}}\NormalTok{n,}\DecValTok{10}\NormalTok{,}\DecValTok{1}\NormalTok{)}
\NormalTok{  result.ImpS[count] }\OtherTok{=} \FunctionTok{mean}\NormalTok{((u}\SpecialCharTok{\textgreater{}}\DecValTok{10}\NormalTok{)}\SpecialCharTok{*}\FunctionTok{exp}\NormalTok{(}\DecValTok{50{-}10}\SpecialCharTok{*}\NormalTok{u))}
\NormalTok{\}}
\FunctionTok{rbind}\NormalTok{(sampleSize.ImpS,}\FunctionTok{signif}\NormalTok{(result.ImpS,}\DecValTok{5}\NormalTok{))}
\end{Highlighting}
\end{Shaded}

\begin{verbatim}
##                       [,1]       [,2]       [,3]       [,4]      [,5]
## sampleSize.ImpS 3.0000e+00 4.0000e+00 5.0000e+00 6.0000e+00 7.000e+00
##                 7.9322e-24 7.3357e-24 7.6384e-24 7.6285e-24 7.637e-24
##                       [,6]
## sampleSize.ImpS 8.0000e+00
##                 7.6162e-24
\end{verbatim}

\end{document}
